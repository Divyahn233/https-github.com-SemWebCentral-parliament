% !TEX encoding = UTF-8 Unicode
% !TEX TS-program = XeLaTeX
% !TEX root = ParliamentUserGuide.tex

%%%%%%%%%%%%%%%%%%%%%%%%%%%%%%%%%%%%%%%%%%%%%%%%%%
\chapter{Building \acl{pmnt}}
\label{chapter-building-parliament}

\ac{pmnt} is a cross-platform, mixed-language library.  It's core is written in portable C++, but it also has a Java interface.  As a result of both the cross-platform and multi-language requirements, the build infrastructure for \ac{pmnt} requires a little bit of work to configure.  This chapter is your guide through that process.

\acp{pmnt} build infrastructure has two main parts.  The top-level portion is based on ant, a build tool widely used in the Java development community.  This portion of the infrastructure builds the Java half of the \ac{pmnt} code base, and it also invokes the second portion, which is based on Boost.Build.  Boost.Build is a system that is well-adapted to building C++ code.  It has the advantages of being portable and much simpler to use than make files.  It is also the standard build system of the Boost project, whose libraries are used by the C++ portion of \ac{pmnt}.

This chapter will step through the libraries and tools that \ac{pmnt} depends upon and show you how to configure them on your system.  At the end of this chapter, you should have a working copy of the \ac{pmnt} source code from which you can build \ac{pmnt} binaries.

%%%%%%%%%%%%%%%%%%%%%%%%%%%%%%%%%%%%%%%%%%%%%%%%%%
\section{Platforms and Prerequisites}

You will need to acquire and install one or more appropriate compilers for each operating system on which you wish to build.  \ac{pmnt} has been tested on the platform and compiler combinations shown in Table~\ref{platforms-and-compilers}.  Note that the last column shows the corresponding Boost.Build toolset name, which will be used extensively in the sections below as we configure the \ac{pmnt} build infrastructure.

\begin{table}[htbp]
	\centering
	\begin{tabular}{lll}
		\toprule
		\textbf{Operating System} & \textbf{Compiler} & \textbf{Toolset} \\
		\headingrule
		Windows 10 (32- and 64-bit) & Visual Studio 2019 & msvc-14.2 \\
		\midrule
		Mac OS X 10.14 (Mojave) & Xcode 11.3.1 & clang \\
		\midrule
		Ubuntu 20 (64-bit) & GCC 9.3.0 & gcc \\
		\midrule
		Centos 7 (64-bit) & GCC 4.8.5 & gcc \\
		\midrule
		Centos 8 (64-bit) & GCC 8.3.1 & gcc \\
		\bottomrule
	\end{tabular}
	\caption{Supported Platforms and Compilers}
	\label{platforms-and-compilers}
\end{table}

\acp{pmnt} capacity is much higher when running as a 64-bit process\footnote{On 32-bit Windows \ac{pmnt} runs out of virtual address space after storing 5 to 10 million statements.}, so 64-bit builds are recommended.

On Macintosh, \ac{pmnt} builds as a universal binary using Apple's Xcode development tools.

\ac{pmnt} assumes the presence of the Java Developer Kit (JDK), version 8.  Furthermore, you will need a 64-bit \ac{jvm} in order to run a 64-bit build of \ac{pmnt}.
\begin{itemize}
	\item On Windows, download and install the JDK from Oracle.  The 32-bit and 64-bit versions are separate downloads and installations.

	\item On Macintosh, download and install the JDK from Oracle.

	\item On Linux, you may need to install one or more packages, depending on your particular distribution.
\end{itemize}

You will need Apache Ant version 1.10.1 or later\urlcite{http://ant.apache.org/} and Apache Ivy version 2.4.0 or later\urlcite{http://ant.apache.org/ivy}.  Install these according to their documentation.  Finally, you need a client for the git version control system\urlcite{https://git-scm.com/downloads}.

Once you have these prerequisites installed, you can clone the \ac{pmnt} code base from here:
\begin{quote}
	\url{https://github.com/SemWebCentral/parliament}
\end{quote}
Because GitHub limits repository size, \acp{pmnt} does not include larger binary files that are required by the build.  To supply these, download this archive:
\begin{quote}\sloppy
	\url{https://github.com/SemWebCentral/parliament/releases/download/dependencies-«latest-date»/parliament-dependencies-«latest-date».zip}
\end{quote}
and expand it in the \path|dependencies| subdirectory of your clone.

%%%%%%%%%%%%%%%%%%%%%%%%%%%%%%%%%%%%%%%%%%%%%%%%%%
\section{Configuring Eclipse}

The \ac{pmnt} code base includes Eclipse projects for all of its Java and C++ code.  These are useful for inspecting and editing code, but it is important to note that they are not the official build mechanism.  In fact, it is difficult to configure the C++ Eclipse projects to build at all.  (The Java projects do build correctly, but they are still not the official build mechanism.)  This may be corrected in the future, but the \ac{jni} layer between the Java and native code makes this complex.  Therefore the C++ projects should be regarded merely as an editing convenience.

To setup Eclipse, you need the Photon or later version of the Eclipse \ac{ide} for Java Developers, plus the \ac{cdt}.  One way to acquire this set of components is to download the Eclipse \ac{ide} for Java\urlcite{http://www.eclipse.org/}, install it, and use its ``Install New Software'' menu item to download and install the \ac{cdt}.  The procedure for this changes a bit between Eclipse releases, but you can find instructions on the \ac{cdt} web site\urlcite{http://www.eclipse.org/cdt/}.

To use Eclipse with your \ac{pmnt} working copy, first choose (or create) an Eclipse workspace.  Then import all existing projects from within your \ac{pmnt} working copy.  To do so, choose Import from the File menu and select ``Existing Projects into Workspace'' under the General category.  Press the Next button, and enter the root directory of your \ac{pmnt} working copy in the ``Select root directory'' box.  Press the Select All button, make sure that ``Copy projects into workspace'' is unchecked, and press the Finish button.  At this point all of the \ac{pmnt} projects will be displayed in the Package Explorer view.

%%%%%%%%%%%%%%%%%%%%%%%%%%%%%%%%%%%%%%%%%%%%%%%%%%
\section{Building Berkeley DB}
\label{sec:BuildingBerkeleyDB}

\ac{pmnt} uses Berkeley DB (often abbreviated BDB), an embedded database manager from Oracle\urlcite{http://www.oracle.com/us/products/database/berkeley-db/}.  Because \ac{pmnt} is open source, this use of Berkeley DB also falls under an open source license.  The following procedures are based on BDB version 5.3.28.

%Note that \ac{pmnt} versions 2.7.6 through 2.7.9 was released with Berkeley DB 6.1, but we have rolled back to 5.3 because Oracle sneakily changed their open source license from a BSD-like license to AGPL, which is too restrictive for our purposes.

\subsection{Building BDB for Windows}

The build infrastructure for Berkeley DB is not particularly friendly for Windows.  Therefore, the \ac{pmnt} dependencies archive includes pre-built Berkeley DB libraries (both 32- and 64-bit).  If you need to update the pre-built libraries, e.g., for a new version of Berkeley DB or to build with a different compiler, see Appendix~\ref{chapter-building-bdb-for-windows}.

Define the following environment variables so that the \ac{pmnt} build infrastructure can find the libraries:
\begin{verbatim}
BDB_VERSION=53
BDB_HOME=«dir»/dependencies/bdb
\end{verbatim}
where \path|«dir»| is the absolute path of your \ac{pmnt} working copy.


\subsection{Building BDB for Macintosh}

On Macintosh, Berkeley DB follows the usual pattern of software based on the autoconf/automake/libtool suite.  Specifically, expand the BDB distribution archive file.  In the file \path|src/dbinc/atomic.h|, replace all instances of \verb|__atomic_compare_exchange| with \verb|__atomic_compare_exchange_db|, \texttt{cd} to the \path|build_unix| subdirectory, and issue the following commands:
\begin{verbatim}
env CC=clang CFLAGS="-fvisibility=default -arch x86_64"
     ../dist/configure --enable-posixmutexes
make
sudo make install
\end{verbatim}
The \verb|CFLAGS| setting above causes the build to produce universal binaries.  You can tidy up after the build with the command \verb|make realclean|.  Once you have built and installed Berkeley DB, define the following environment variables so that the \ac{pmnt} build infrastructure can find the libraries:
\begin{verbatim}
BDB_VERSION=5.3
BDB_HOME=/usr/local/BerkeleyDB.5.3
\end{verbatim}

\subsection{Building BDB for Linux}

On Linux, Berkeley DB follows the usual pattern of software based on the autoconf/automake/libtool suite.  Specifically, expand the BDB distribution archive file.  In the file \path|src/dbinc/atomic.h|, replace all instances of \verb|__atomic_compare_exchange| with \verb|__atomic_compare_exchange_db|, \texttt{cd} to the \path|build_unix| subdirectory, and issue the following commands:
\begin{verbatim}
env CFLAGS="-m64" ../dist/configure
make
sudo make install
\end{verbatim}
You can tidy up after the build with the command \verb|make realclean|.  Once you have built and installed Berkeley DB, define the following environment variables so that the \ac{pmnt} build infrastructure can find the libraries:
\begin{verbatim}
BDB_VERSION=5.3
BDB_HOME=/usr/local/BerkeleyDB.5.3
\end{verbatim}

%%%%%%%%%%%%%%%%%%%%%%%%%%%%%%%%%%%%%%%%%%%%%%%%%%
\section{Building the Boost Libraries}
\label{sec:BuildingBoost}

Since the Boost project is unfamiliar to many, here is an introduction taken from the Boost web site:\urlcite{http://boost.org/}
\begin{quote}\small
Boost provides free peer-reviewed portable C++ source libraries.

We emphasize libraries that work well with the C++ Standard Library. Boost libraries are intended to be widely useful, and usable across a broad spectrum of applications. The \href{http://www.boost.org/users/license.html}{Boost license} encourages both non-commercial and commercial use.

We aim to establish ``existing practice'' and provide reference implementations so that Boost libraries are suitable for eventual standardization. Ten Boost libraries are already included in the \href{http://www.open-std.org/jtc1/sc22/wg21/}{C++ Standards Committee's} Library Technical Report (\href{http://www.open-std.org/jtc1/sc22/wg21/docs/papers/2005/n1745.pdf}{TR1}) and in the new C++11 Standard.  C++11 also includes several more Boost libraries in addition to those from TR1.  More Boost libraries are proposed for standardization in C++17.
\end{quote}

To get started, download the Boost source distribution (version 1.74.0 or later) from the Boost web site.  Unpack this to a handy location on your disk.  This location may be anywhere you like.  Define this environment variable pointing there:

{
	%\renewcommand{\arraystretch}{1.5}
	\renewcommand{\tabcolsep}{0pt}
	\begin{tabular}{l@{\hspace{2em}}l}
		\verb|set BOOST_ROOT=C:\boost_1_74_0|
			& (on Windows)\\
		\verb|export BOOST_ROOT=~/boost_1_74_0|
			& (on Macintosh and Linux)\\
	\end{tabular}
}

From \acp{pmnt} point of view, there are two primary components contained within \verb|BOOST_ROOT|.  The first (and most obvious) is the boost libraries themselves.  Most of these are so-called ``header-only'' libraries, meaning that there is no pre-compiled library code.  All of the code of such libraries is referenced via \verb|#include| directives and compiled along with the calling code.  Such libraries are extremely convenient, because they require little setup.  \ac{pmnt} also uses several Boost libraries that are not header-only.  We will discuss how to build these libraries below.

The second major Boost component is Boost.Build.  This is a cross-platform build system (located in the \path|BOOST_ROOT/tools/build|) that is written in a specialized interpreted language whose interpreter is a command called \verb|b2|.  The Boost community does not provide \verb|b2| binaries.  Rather, the Boost distribution contains a bootstrapping script that builds \verb|b2| from source on your platform.

To build the minimal set of libraries required for \ac{pmnt}, follow the directions below for your platform.  In each case, you will invoke the bootstrap script to create the Boost.Build interpreter, and then you will invoke that to build the libraries.

\subsection{Building Boost on Windows}

Open a Command Prompt, change to the \verb|BOOST_ROOT| directory, and issue the command ``\verb|.\bootstrap.bat|''.  This will build the executable \verb|b2.exe| in \verb|BOOST_ROOT|.  Move this binary to any location on your path.  Then issue the following command:

{\small\begin{verbatim}
b2 -q -j3 --disable-icu --ignore-site-config --layout=versioned
--build-dir=build-msvc --stagedir=stage-msvc --with-atomic
--with-chrono --with-date_time --with-filesystem --with-log
--with-regex --with-test --with-thread toolset=msvc
define=BOOST_TEST_ALTERNATIVE_INIT_API address-model=32,64
variant=debug,release link=shared,static runtime-link=shared stage
\end{verbatim}}

This will build the libraries in \path|stage-msvc/lib|.  The following commands will delete intermediate build products to save disk space:

{\small\begin{verbatim}
del bjam.exe bootstrap.log project-config.jam
\end{verbatim}}

{\small\begin{verbatim}
rd /s/q build-msvc tools\build\src\engine\bin.ntx86
tools\build\src\engine\bootstrap
\end{verbatim}}

\subsection{Building Boost on Macintosh}

Open a Terminal window, change to the \verb|BOOST_ROOT| directory, and issue the following command:

{\small\begin{verbatim}
./bootstrap.sh --without-icu --with-toolset=clang --with-libraries=
atomic,chrono,date_time,filesystem,log,regex,test,thread
\end{verbatim}}

This will build the executable \verb|b2| in \verb|BOOST_ROOT|.  Move this binary to any location on your path.  Then issue the following command:

{\small\begin{verbatim}
b2 -q -j3 --disable-icu --ignore-site-config --layout=versioned
--build-dir=build-clang --stagedir=stage-clang toolset=clang
define=BOOST_TEST_ALTERNATIVE_INIT_API address-model=32_64
variant=debug,release link=shared,static runtime-link=shared
cxxflags="-std=c++17" linkflags="-std=c++17" stage
\end{verbatim}}

This will build the libraries in \path|stage-clang/lib|.  The following commands will delete intermediate build products to save disk space:

{\small\begin{verbatim}
rm -r bjam bootstrap.log project-config.jam build-clang/
tools/build/src/engine/bootstrap/ tools/build/src/engine/bin.*
\end{verbatim}}

\subsection{Building Boost on Linux}

At a shell, change to the \verb|BOOST_ROOT| directory, and issue the following command:

{\small\begin{verbatim}
./bootstrap.sh --without-icu --with-toolset=gcc --with-libraries=
atomic,chrono,date_time,filesystem,log,regex,test,thread
\end{verbatim}}

This will build the executable \verb|b2| in \verb|BOOST_ROOT|.  Move this binary to any location on your path.  Then issue the following command:

{\small\begin{verbatim}
b2 -q -j3 --disable-icu --ignore-site-config --layout=versioned
--build-dir=build-gcc --stagedir=stage-gcc toolset=gcc
define=BOOST_TEST_ALTERNATIVE_INIT_API address-model=64
variant=debug,release link=shared,static runtime-link=shared
cxxflags=-std=c++17 linkflags=-std=c++17 stage
\end{verbatim}}

This will build the libraries in \path|stage-gcc/lib|.  If your compiler does not support the C++17 standard, change ``17'' to ``14'' or ``11'' in the above command (in two places).  The following commands will delete intermediate build products to save disk space:

{\small\begin{verbatim}
rm -r bjam bootstrap.log project-config.jam build-gcc/
tools/build/src/engine/bootstrap/ tools/build/src/engine/bin.*
\end{verbatim}}


%%%%%%%%%%%%%%%%%%%%%%%%%%%%%%%%%%%%%%%%%%%%%%%%%%
\section{Configuring Boost.Build}

Next we need to configure Boost.Build so that it can build \ac{pmnt}.  Start by defining the following environment variables:
\begin{itemize}
	\item\verb|JAVA_HOME|: The location of your JDK installation, which must be version 8 or higher.

	\item\verb|BDB_VERSION|: As described above in Section~\ref{sec:BuildingBerkeleyDB}.

	\item\verb|BDB_HOME|: As described above in Section~\ref{sec:BuildingBerkeleyDB}.

	\item\verb|BOOST_VERSION|: The version of Boost (currently 1\_74\_0).

	\item\verb|BOOST_ROOT|: As described above in Section~\ref{sec:BuildingBoost}.

	\item\verb|BOOST_BUILD_PATH|: The sub-directory \path|tools/build| of \verb|BOOST_ROOT|.

	\item\verb|BOOST_TEST_LOG_LEVEL|: Controls the output volume of the C++ unit tests.  Possible values and their meanings are listed in Table~\ref{boost-test-log-level-values}.

	\begin{table}[htbp]
		\centering
		\begin{tabular}{ll}
			\toprule
			\textbf{Value} & \textbf{Meaning} \\
			\headingrule
			\verb|all|           & report all log messages \\
			\verb|success|       & the same as all \\
			\verb|test_suite|    & show test suite messages \\
			\verb|message|       & show user messages \emph{(useful default)} \\
			\verb|warning|       & report warnings issued by user \\
			\verb|error|         & report all error conditions \\
			\verb|cpp_exception| & report uncaught C++ exceptions \\
			\verb|system_error|  & report system-originated non-fatal errors \\
			\verb|fatal_error|   & report only fatal errors \\
			\verb|nothing|       & do not report any information \\
			\bottomrule
		\end{tabular}
		\caption{Possible Values of \texttt{BOOST\_TEST\_LOG\_LEVEL}}
		\label{boost-test-log-level-values}
	\end{table}

	\item\verb|VS_ROOT|: \emph{(Windows only)} The root of your Visual Studio installation.  Usually \path|C:\Program Files (x86)\Microsoft Visual Studio\2019\Community\VC\|.

	\item\verb|VS_BAT_PATH_32|: \emph{(Windows only)} The path to the batch file that sets up the environment for 32-bit builds.  Typically this is \path|%VS_ROOT%Auxiliary\Build\vcvarsamd64_x86.bat|.

	\item\verb|VS_BAT_PATH_64|: \emph{(Windows only)} The path to the batch file that sets up the environment for 64-bit builds.  Typically this is \path|%VS_ROOT%Auxiliary\Build\vcvars64.bat|.

	\item\verb|VS_REDIST_PATH_32|: \emph{(Windows only)} The path to the 32-bit redistributables installer.  Typically this is \path|%VS_ROOT%Redist\MSVC\14.27.29016\vc_redist.x86.exe|.

	\item\verb|VS_REDIST_PATH_64|: \emph{(Windows only)} The path to the 64-bit redistributables installer.  Typically this is \path|%VS_ROOT%Redist\MSVC\14.27.29016\vc_redist.x64.exe|.
\end{itemize}

Next we create two Boost.Build configuration files, \path|site-config.jam| and \path|user-config.jam|.  Boost.Build reads these files on startup.  The two are separate so that the first one can be installed and maintained by a system administrator, and the second by the individual user.  These files can be placed in a number of locations. Table~\ref{boost-build-config-search} explains where Boost.Build searches to find these files.

\begin{table}[htbp]
	\centering
	\begin{tabular}{>{\small}l>{\RaggedRight\small}p{49mm}>{\RaggedRight\small}p{49mm}}
		\toprule
		OS & \path|site-config.jam| & \path|user-config.jam| \\
		\headingrule
		Unix-like:
			&	\path|/etc| \newline
				\verb|$HOME| \newline
				\verb|$BOOST_BUILD_PATH|
			&	\verb|$HOME| \newline
				\verb|$BOOST_BUILD_PATH| \\
		\midrule
		Windows:
			&	\verb|%SystemRoot%| \newline
				\verb|%HOMEDRIVE%%HOMEPATH%| \newline
				\verb|%HOME%| \newline
				\verb|%BOOST_BUILD_PATH%|
			&	\verb|%HOMEDRIVE%%HOMEPATH%| \newline
				\verb|%HOME%| \newline
				\verb|%BOOST_BUILD_PATH%| \\
		\bottomrule
	\end{tabular}
	\caption{Boost.Build Search Paths for Configuration Files}
	\label{boost-build-config-search}
\end{table}

Some people prefer to keep these files together with their Boost.Build installation, placing them in \verb|BOOST_BUILD_PATH|.  There are example \path|site-config.jam| and \path|user-config.jam| files in that directory, and so you will have to replace (or rename) them if you choose this option.  Others prefer to separate the configuration files from the Boost.Build installation so that they can update to a new version of Boost.Build without having to first save \path|site-config.jam| and \path|user-config.jam| and then restore them after the update is complete.  On Windows, using the \verb|HOME| location requires defining the environment variable \verb|HOME|, because Windows does not define it by default.

Within your working copy of the \ac{pmnt} repository, in the directories \path|doc/MacOS|, \path|doc/Windows|, and \path|doc/Linux|, you will find example configuration files for Macintosh, Windows, and Linux respectively that you can copy and customize.  The most important customization that you need to make is to remove (or comment out) any lines in \path|user-config.jam| for compiler versions that you have not installed.

%%%%%%%%%%%%%%%%%%%%%%%%%%%%%%%%%%%%%%%%%%%%%%%%%%
\section{Building \ac{pmnt} Itself}

You are now ready to build \ac{pmnt}.  To do so, issue the command \path|ant| from the root directory of your \ac{pmnt} working copy --- \emph{but don't try this until you have read the next couple of paragraphs.}  This command builds the entire repository, including both native and Java code, and create a distribution-ready package in this directory:
\begin{quote}
	\texttt{targets/Parliament-v\textbf{\textit{X.Y.Z}}-\textbf{\textit{toolset}}}
\end{quote}
where \texttt{\textbf{\textit{X.Y.Z}}} is the \ac{pmnt} version number and \texttt{\textbf{\textit{toolset}}} indicates the platform on which the distribution runs, as shown in Table~\ref{tbl:ToolsetPlatformCorrespondence}.  The \path|ant clean| command deletes all build products.

The file \path|build.properties|, located in the root of your working copy, controls the \ac{pmnt} build.  This file does not exist by default.  If the build does not find it, it uses \path|build.properties.default| instead.  The latter contains the build options used to create an official release of \ac{pmnt}, which typically includes several different release builds.  To build a single variant, copy \path|build.properties.default| to \path|build.properties| and then customize the latter.  The file itself contains instructions.  Please change \path|build.properties.default| directly \emph{only if you intend to update the official release build options.}

The \path|build.properties| file also contains an option that disables the native code unit tests, \verb|skipNativeUnitTest|.  This is important when cross-compiling, e.g., building 64-bit binaries on 32-bit Windows.  If you use this option, be sure to change it only in \path|build.properties|, not \path|build.properties.default|.

For more targeted builds, ant can be run from many sub-directories in your working copy.  Here is a road map to the various sub-projects:
\begin{itemize}
	\item\path|Parliament|: The native code at the heart of \ac{pmnt}, plus its \ac{jni} interface
	\item\path|jena/JosekiParliamentClient|: A client-side Java library for communicating with a Joseki-\ac{pmnt} \ac{sparql} endpoint
	\item\path|jena/JenaGraph|: Enables Jena to use \ac{pmnt} for storing models
	\item\path|jena/JosekiExtensions|: Extensions to Joseki that, together with JenaGraph, create a \ac{sparql} endpoint on top of \ac{pmnt}
	\item\path|jena/SpatialIndexProcessor|: A JenaGraph add-on for processing spatial queries efficiently
	\item\path|jena/TemporalIndexProcessor|: A similar add-on to speed up temporal queries
\end{itemize}

When working on the native code portions of \ac{pmnt}, it can be useful to run the \path|b2| portion of the build directly.  To do so, change directory to \path|KbCore| (for the \ac{pmnt} \ac{dll} itself), \path|AdminClient| (for the command line interface to \ac{pmnt}), or \path|Test| (for the unit tests).  These directories are located within the \path|Parliament| sub-directory of your working copy.  Then issue the command
\begin{verbatim}
b2 -q «build-options»
\end{verbatim}
Here \verb|«build-options»| is a placeholder for one set of build options from \path|build.properties|, described above.  The \verb|-q| option causes \path|b2| to quit immediately whenever an error occurs, so that you do not have to scroll up through the build output to verify that the build was successful.
